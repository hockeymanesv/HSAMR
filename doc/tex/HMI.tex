\documentclass[../documentation.tex]{subfiles}

%Dokumententitel, Autor und Datum
\title{Human Maschine Interface}
\author{Paul Förster}
\date{\today{}, Dresden}

\begin{document}
%Titelseite und Inhaltsverzeichnis für alleinstehende Modul-PDF
\onlyinsubfile{\maketitle\tableofcontents}
%Wichtig: Abschnitt für documentation.tex
\section{Human Maschine Interface}

%Ab hier eure Spielwiese
%Erstes Kapitel - Nummerierung erfolgt automatisch
\subsection{Anforderungen}
\label{sec:anforderungen}

\begin{itemize}
\item Roboter lässt sich zu jeder Zeit anhalten
\item Parcour als Karte in Vogelperspektive
\item Position des Roboters auf Karte
\item gefundene Parklücken auf Karte
\item Darstellung aller erfassten Messwerte
\item Parken
\begin{itemize}
\item gezielt in ausgewählte Parklücke
\item sofort in nächst beste Parklücke
\end{itemize}
\item Ausparken soll automatisch mit scout-Modus aktiviert werden
\end{itemize}

\subsection{Material-Design unter Android 4}
\label{sec:materialdesignunterandroid4}

Material Design wurde erst mit Android 5 unter der API 25 eingeführt. Damit wir Material Design nutzen können, müssen Support Libraries nach installiert werden.

Ab hier durch klicken https://developer.android.com/training/basics/supporting-devices/platforms.html

Die Installation der Bibliotheken erwies sich als durchaus problematisch, da Eclipse ewig nicht mehr unterstützt wird.
Ich habe folgendes Repo geklont https://github.com/koush/android-support-v7-appcompat und importiert. Dann als "is Library" markiert und in mein Projekt importiert.

Hat alles nicht geklapt. Bin jetzt auf Android Studio umgestiegen. Dafür habe ich die NXT Datein mit einem Hardlink in das AndroidStudioProjekt gepackt /usr/local/Cellar/hardlink-osx/0.1.1/bin/hln ~/Documents/code/HSAMR/NXT/src/parkingRobot/ ~/AndroidStudioProjects/NXT\ Fernsteuerung/nXT/src/main/java/parkingRobot




\end{document}